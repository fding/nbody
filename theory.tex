\documentclass[11pt]{article}
\setlength{\topmargin}{-0.5in}
\setlength{\textheight}{8.5in}
\setlength{\oddsidemargin}{0in}
\setlength{\evensidemargin}{0in}
\setlength{\textwidth}{7in}

\usepackage{amsmath,amssymb,amsthm}
\usepackage[pdftex]{graphicx}
\setlength{\headheight}{15.2pt}

\newtheorem{theorem}{Theorem}
\newtheorem{lemma}{Lemma}
\newtheorem{definition}{Definition}
\newtheorem{proposition}{Proposition}
\setlength{\parskip}{1ex plus 0.5ex minus 0.2ex}

\title{Various integrators}
\author{Fengning Ding}

\begin{document}
\maketitle

\section{Numerical solutions of ordinary differential equations}
Of the below integrators, Euler's method, the improved Euler's method,and the Runge-Kunta 4 method 
are general integrators. The symplectic integrators solve a Hamiltonian system (such as Newton's second law without nonconservative forces)
better, since they conserve energy, momentum, angular momenum, etc., better.
This is because symplectic integrators solve a nearby Hamiltonian exactly. All energy losses are due to round-off error.
All methods require a step-size, $h$.
\subsection{Euler's Method}
This is a first order nonsymplectic method.
\begin{verbatim}
1. Find a(x[i],v[i],t)
2. Set x[i+1]=x[i]+v[i]*h
3. Set v[i+1]=v[i]+a[i]*h.
\end{verbatim}

\subsection{Improved Euler's Method}
With little extra work, we can increase the accuracy (to order 2) by using
\begin{verbatim}
1. Find a(x[i],v[i],t)
2. Set x[i+1]=x[i]+v[i]*h+0.5*a[i]*h*h
3. Set v[i+1]=v[i]+a[i]*h.
\end{verbatim}

\subsection{Runge-Kunta 4}
This is a fourth order nonsymplectic method.
\begin{verbatim}
1. Find a(x[i],v[i],t)
2. Set k1=h*a, j1=h*v
3. Find a(x[i]+0.5*k1,v[i]+0.5*j1,t+0.5*h)
4. Set k2=h*a, j2=h*v
5. Find a(x[i]+0.5*k2,v[i]+0.5*j2,t+0.5*h)
6. Set k3=h*a, j3=h*v
7. Find a(x[i]+k3,v[i]+j3,t+h)
8. Set k4=h*a, j4=h*v
9. Set x[i+1]=x[i]+1/6*(j1+2*j2+2*j3+j4)
10. Set v[i+1]=v[i]+1/6*(k1+2*k2+2*k3+k4).
\end{verbatim}
\subsection{Yoshida Methods (and Verlet)}
The Yoshida methods are a general set of symplectic integrators of even order.
The general outline is
\begin{verbatim}
1. Set x=x[i], v=v[i]
2. For i=1 to n-1:
    2a. x=x+0.5*h*p[i]*v
    2b. Find a(x,v)
    2c. v=v+0.5*h*p[i]*a
    2d. x=x+0.5*h*p[i]*v
3. x=x+0.5*h*p[n]*v
4. Find a(x,v)
5. v=v+0.5*h*p[n]*a
6. x=x+0.5*h*p[n]*v
7. For i=n-1 to 1:
    7a. x=x+0.5*h*p[i]*v
    7b. Find a(x,v)
    7c. v=v+0.5*h*p[i]*a
    7d. x=x+0.5*h*p[i]*v
8. x[i+1]=x
9. v[i+1]=v.
\end{verbatim}
For the Yoshida order 2 method 
(otherwise known as the Verlet method), $n=1$ and
$p_1=1$.

For the Yoshida order 4 method, $n=2$ and
$p_1=1.351207191959657$, $p_2=-1.702414383919315$.

For the Yoshida order 6 method, $n=4$ and\\
\begin{eqnarray*}
p_1&=&0.784513610477560\\
p_2&=&0.235573213359357\\
p_3&=&-1.17767998417887\\
p_4&=&1.31518632068391.
\end{eqnarray*}

For the Yoshida order 8 method, $n=8$ and\\
\begin{eqnarray*}
p_1&=&1.04242620869991\\
p_2&=&1.82020630970714\\
p_3&=&0.157739928123617\\
p_4&=&2.44002732616735\\
p_5&=&-0.00716989419708120\\
p_6&=&-2.44699182370524\\
p_7&=&-1.61582374150097\\
p_8&=& -1.7808286265894516.
\end{eqnarray*}

\subsection{Ruth Method}
This is a third order symplectic integrator.


Let \verb|p=[2/3,-2/3,1]| and \verb|q=[7/24,3/4,-1/24]|
\begin{verbatim}
1. Set x=x[i], v=v[i]
2. For i=1 to 3:
    2a. Find a(x,v)
    2b. Set v=v+p[i]*a*h
    2c. Set x=x+q[i]*v*h
3. Set x[i+1]=x
4. Set v[i+1]=v.
\end{verbatim}

\section{Forest-Ruth}
This is a fourth order symplectic integrator.

Let $f_1=1.35120719195966$, $f_2=1-f_1$, $f_3=1-2*f_1$.
\begin{verbatim}
1. Set x=x[i], v=v[i]
2. Set x=x+f1*v*h/2
3. Find a(x,v)
4. Set v=v+f1*a*h
5. Set x=x+f2*v*h/2
6. Find a(x,v)
7. Set v=v+f3*a*h
8. Set x=x+f2*v*h/2
9. Find a(x,v)
10. Set v[i]=v+f1*a*h
11. Set x[i]=x+f1*v*h/2.
\end{verbatim}

\subsection{Position Extended Forest Ruth Method}
This is another fourth order Forest-Ruth algorithm.

Let $f_1=0.1786178958448091$, $f_2=-0.2123418310626054$,
$f_3=-0.0662645266981849$.
\begin{verbatim}
1. Set x=x[i], v=v[i]
2. Set x=x+f1*v*h
3. Find a(x,v)
4. Set v=v+(1-2*f2)*a*h/2
5. Set x=x+f3*v*h
6. Find a(x,v)
7. Set v=v+f2*a*h
8. Set x=x+(1-2*f3-2*f1)*v*h
9. Find a(x,v)
10. Set v=v+f2*a*h
11. Set x=x+f3*v*h
12. Find a(x,v)
13. Set v[i+1]=v+(1-2*f2)*a*h/2
14. Set x[i+1]=x+f1*v*h.
\end{verbatim}

\subsection{Candy Rozmus}
This is a fourth-order symplectic method.

Let
$p_1=p_4=\frac{1}{2(2-2^{1/3})}$,
$p_2=p_3=\frac{1-2^{1/3}}{2(2-2^{1/3})}$,
$q_1=0$,
$q_2=q_4=\frac{1}{2-2^{1/3}}$,
and
$q_3=-\frac{2^{1/3}}{2-2^{1/3}}$.
\begin{verbatim}
1. Set x=x[i], v=v[i]
2. For i=1 to 4:
    2a. Find a(x,v)
    2b. v=v+q[i]*a*h
    2c. x=x+p[i]*v*h
3. x[i+1]=x
4. v[i+1]=v.
\end{verbatim}
We could also optimize this algorithm by setting
\begin{eqnarray*}
p_1&=&0.5153528374311229364\\
p_2&=&-0.085782019412973646\\
p_3&=&0.4415830236164665242\\
p_4&=&0.1288461583653841854\\
q_1&=&0.1344961992774310892\\
q_2&=&-0.2248198030794208058\\
q_3&=&0.7563200005156682911\\
q_4&=&0.3340036032863214255.
\end{eqnarray*}

\section{Major factors on integration accuracy}

\begin{thebibliography}{9}
\bibitem{1}
Rodney Dunning, \emph{VPNBody---Solar-System Dynamics for VPython}.\\ http://www.longwood.edu/staff/dunningrb/vpnbody/
%\bibitem{2}
%Donald Knuth, \emph{The Art of Computer Programming}, Books 1 and 2. Addison Wesley Longman, 1997, Edition 3.
\end{thebibliography}
\end{document}




